% !TEX TS-program = pdflatex
% !TEX encoding = UTF-8 Unicode

% This is a simple template for a LaTeX document using the "article" class.
% See "book", "report", "letter" for other types of document.

\documentclass[20pt]{article} % use larger type; default would be 10pt

\usepackage[utf8]{inputenc} % set input encoding (not needed with XeLaTeX)

%%% Examples of Article customizations
% These packages are optional, depending whether you want the features they provide.
% See the LaTeX Companion or other references for full information.

%%% PAGE DIMENSIONS
\usepackage{geometry} % to change the page dimensions
\geometry{a4paper} % or letterpaper (US) or a5paper or....
% \geometry{margin=2in} % for example, change the margins to 2 inches all round
% \geometry{landscape} % set up the page for landscape
%   read geometry.pdf for detailed page layout information

\usepackage{graphicx} % support the \includegraphics command and options

% \usepackage[parfill]{parskip} % Activate to begin paragraphs with an empty line rather than an indent

%%% PACKAGES
\usepackage{booktabs} % for much better looking tables
\usepackage{array} % for better arrays (eg matrices) in maths
\usepackage{paralist} % very flexible & customisable lists (eg. enumerate/itemize, etc.)
\usepackage{verbatim} % adds environment for commenting out blocks of text & for better verbatim
%\usepackage{subfig} % make it possible to include more than one captioned figure/table in a single float
\usepackage{mathtools}
\usepackage{graphicx} % supports images in latex
% These packages are all incorporated in the memoir class to one degree or another...

\usepackage{graphicx}
\usepackage{subcaption}

%%% Other stuff
\DeclarePairedDelimiter\ceil{\lceil}{\rceil}
\DeclarePairedDelimiter\floor{\lfloor}{\rfloor}

%%% HEADERS & FOOTERS
\usepackage{fancyhdr} % This should be set AFTER setting up the page geometry
\pagestyle{fancy} % options: empty , plain , fancy
\renewcommand{\headrulewidth}{0pt} % customise the layout...
\lhead{}\chead{}\rhead{}
\lfoot{}\cfoot{\thepage}\rfoot{}

%%% SECTION TITLE APPEARANCE
\usepackage{sectsty}
\allsectionsfont{\sffamily\mdseries\upshape} % (See the fntguide.pdf for font help)
% (This matches ConTeXt defaults)

%%% ToC (table of contents) APPEARANCE
\usepackage[nottoc,notlof,notlot]{tocbibind} % Put the bibliography in the ToC
\usepackage[titles,subfigure]{tocloft} % Alter the style of the Table of Contents
\renewcommand{\cftsecfont}{\rmfamily\mdseries\upshape}
\renewcommand{\cftsecpagefont}{\rmfamily\mdseries\upshape} % No bold!

%%% graphics path


%%% END Article customizations

%%% nice things to keep around
%\begin{figure}[!htbp]
%  	\centering
%   	\begin{subfigure}[p]{0.5\linewidth}
%    	\includegraphics[width=\linewidth]{}
%	\caption{figure 1}
%	\label{fig:sub1}
%   	\end{subfigure}
%\end{figure} 

% \noindent\rule{2cm}{0.4pt} 
%%% puts a small horizontal line

% \mathcal{O} 
%%% big O notation

%%% The "real" document content comes below...

\title{Research Topic Intro}
\author{Liam Dillingham}
%\date{} % Activate to display a given date or no date (if empty),
         % otherwise the current date is printed 

\begin{document}
\maketitle

The mandelbrot set is a set of numbers whose orbits remain constrained for all iterations give their initial parameters. I want to start by clarifying what an orbit is.  

The orbit of a point $x_0$ under some function $f$ is the sequence of points following $x_0$ under iteration, where $x_0$ is called the \textit{seed}. so 
$$x_0, f(x_0)=x_1, f(x_1)=x_2, ..., f(x_{n-1})=x_n$$.

So we are taking the seed and iterating it, looping the output back into the next iteration. We can label the functions iteration index as follows:

$$f^{n}(...f^{2}(f^{1}(x_0))...)$$

for $n$ iterations.  Since we are interested in the mandelbrot set, we can introduce its function and try to analyze the behavior of some of its orbits.  The equation is:

$$f_c^{n}(z) = z^{2} +c$$

Where $c$ is a given parameter.  It is worth noting that fixed orbits of a function are points such that $f^{n}(x_0)=x_0$ for all $n$.  Let's graph our function.  It is a quadratic function with y-intercept at $c$. To find the fixed points of our function, we super-impose the line $f(x)=x$, and the intersections are the fixed orbits.  What happens when we choose a random seed given our parameter $c$? 

$$ f_c(x_0)=x_0^{2}+c$$

First, we start at our seed, and move vertically to the point on the graph, $(x_0, f(x_0))$.  Then we move horizontaly to our diagonal line to the point $(f(x_0),f(x_0)$.  This places us above the point on the quadratic, $(f(x_0), f^{2}(x_0))$, and we can move down to locate it.  By continuing this process, we can find the orbit of our seed $x_0$. \\ 

However, choosing certain seeds, and following the process will result in our orbit exploding to infinity.  Especially if there are no intersections between the line and the curve. \\ 

So the Mandelbrot set is a set within the complex plane, which, instead of choosing random seeds, we choose the seed of 0, and change our parameter $c$ to test every point in the complex plane.  If the orbit of 0 explodes to infinity given a particular $c$, then that point is not in the Mandelbrot set.


\end{document}